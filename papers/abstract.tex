\documentclass[10pt, a4paper]{article}
\usepackage{lrec2006}
\usepackage{graphicx}
\usepackage{url}
\usepackage[hidelinks]{hyperref}
\interfootnotelinepenalty=10000

\title{Guampa: a Toolkit for Collaborative Translation}

\author{Alex Rudnick, Taylor Skidmore, Michael Gasser \\
        Indiana University, School of Informatics and Computing \\
        Bloomington, IN, USA\\
        {\tt \{alexr,taylskid,gasser\}@indiana.edu}}
\date{}

\begin{document}

%%\maketitleabstract
\maketitle
\thispagestyle{empty}

\section{Introduction}
For most of the world's language pairs, large bitext corpora are not readily
available and would be difficult to construct. However, for some language
pairs, not only are there many speakers of both languages, there is a community
of activists dedicated to the continued vitality of their heritage language. In
many of these cases, these speaker/activists recognize that there is a shortage
of written material in their heritage language and that translation from other
languages can help to address this problem. Thus they are often aware of the
contribution that MT would make to their task and are eager to do their part in
creating the bitext corpora that are required for SMT. At the same time, they
know that in the absence of MT systems, it is up to the bilingual speakers
themselves to perform the required translations. This is a daunting task for a
small community, however, and collaborative translation can speed up the
process.

In such contexts it thus makes sense to consider a tool that would facilitate
collaborative translation as well as the gathering of translation examples for
a corpus. We know of no user-friendly FOSS software for collaboratively
translating documents on the web, at least not with an eye towards training MT
systems. We address this perceived need with Guampa, a free software package
for the online collaborative translation of documents. It is meant to help both
language activist/learner communities in generating resources for their
heritage languages and MT researchers in building bitext corpora. We are
especially interested in the common case of language pairs in which one
language (normally the source for translation) has substantial resources but
the other (normally the target) does not. Guampa includes tools for importing
source language articles from Wikipedia as well as exporting bitext suitable
for training MT systems.

\section{Spanish and Guarani in Paraguay}
Our group is particularly interested in building a larger bitext corpus for the
Spanish-Guarani language pair. Spanish and Guarani are the co-official
languages of Paraguay, where most people speak Guarani as their first language
and many are bilingual. Guarani suffers not only from a serious lack of
written materials but also from a neglect in all aspects of public life in
Paraguay. To combat this neglect and save the language from what seems to many
its inevitable decline, a small but very active group of bilinguals has come
together in various forums. Among other things, these activists produce new
written materials in Guarani and bilingual materials in Spanish and Guarani.
They are well aware of the importance of language technology and of translation
to their mission.

Though we focus on Spanish and Guarani, there are many other comparable
language pairs around the world, e.g., English/Telugu (India), French/Wolof
(Senegal), Portuguese/Umbundu (Angola), Chinese/Zhuang (China) and
Russian/Tatar (Russia). Our goal is for Guampa to be useful for researchers and
activists working on any such language pair; anyone can download the Guampa
software and run their own instance for their own purposes.

\section{Related Work}
There is a wealth of software, both open-source and proprietary, to assist
translators in their work, but we are not aware of computer-assisted
translation systems that are designed specifically for collaboration among
non-professional translators. Tatoeba \cite{tatoeba} is a
project dedicated to collecting translations of sentences in many languages.
Users may add translations or correct the translations of other users by
providing alternate translations. However, genuine collaboration is not
facilitated since no history is available, and the focus is on the sentence
rather than the document. Traduwiki \cite{traduwiki} comes closer
to our goals; while it is intended for the collaborative translation of
documents, it is not open-source, and does not offer an easy way to export
training data for machine translation systems. In addition, development appears
to have stopped in 2007.

There have been a number of projects focused on using crowdsourcing to produce
bitext corpora for training MT systems. Notably, Ambati
\cite{ambati_naacl,ambati_act} has used active learning to construct a corpus
for training English-Spanish SMT, automatically creating Mechanical Turk tasks
to elicit translations for phrases that his system did not know how to
translate, but should. The Joshua team at Johns Hopkins has also successfully
used Mechanical Turk to crowdsource the creation of bitext corpora for SMT for
many languages of the Indian subcontinent
\cite{post-callisonburch-osborne:2012:WMT}. Both of these projects relied on a
large population of MTurk users familiar with the source and target languages;
they may not be applicable to languages spoken by smaller populations. We posit
that in this case, supporting an online community of translators would be more
appropriate than one-off Mechanical Turk tasks.

\section{Features of Guampa}
At its core, Guampa is a tool for translating documents. The central interface of
Guampa shows a document's source language text, segmented by sentence,
alongside the current translation for each sentence. For each sentence in the
document, users can add a new translation or edit the current translation. The
software stores the complete history of translation edits, along with comments
on the translation of a given sentence. Users can discuss the
best way to translate a particular passage and see the history of the proposed
translations for a sentence. Thus, Guampa is much like a wiki for translations;
quality control is performed through community consensus.


Like the interface of TraduWiki, but in contrast to that of popular
internationalization tools like Pootle, our layout is intended to be suitable
for reading documents online. We intend it to be helpful for language learners
as well -- a user familiar with the source language and learning the target
language (or vice-versa) might benefit from reading translations side-by-side,
as in dual-language books.

Users will primarily select a document to read or translate on the navigation
interface. Here one can browse the available documents by tag. Additional
sorting criteria, such as recent activity and completeness of translation, will
be added soon.

Users must be logged in to add or edit translations, or to add comments. They
may log in with social media accounts (using OpenID or OAuth), such as Google,
Facebook, or Twitter. Optionally, users can also create Guampa-specific
accounts protected by passwords.

For easy adaptability to different language pairs, the interface is built with
an internationalization package so that its strings can easily be replaced; our
development version has interfaces in English and Spanish, with Guarani coming
soon. Adding more languages as appropriate is straightforward, requiring very
few code changes. Additionally, the sentence segmenter can easily be changed
for locating sentence boundaries in different languages; our development
version uses the Punkt segmenter for Spanish from NLTK \cite{nltkbook}.


\subsection{Importing Documents, Exporting Bitext}
For populating a new instance of Guampa, we include scripts to extract
plain-text versions of Wikipedia articles from Wikipedia database dumps
\footnote{Complete copies of Wikipedias can be downloaded at \\
\url{http://dumps.wikimedia.org/}}, providing source-language documents for a
new installation of Guampa. Our tool builds on the Wikipedia Extractor
script from the Medialab at the University of Pisa \cite{pisa-wp-extractor}.

However, for some source languages, such as English or Spanish, importing an
entire Wikipedia would overwhelm both the server and the users. Fortunately,
many Wikipedias include a list of so-called "vital articles", subjects for
which it is felt that high-quality articles are essential in any encyclopedia.
These lists typically contain roughly one thousand articles. We provide scripts
to extract and import only these articles, and to tag them with their
appropriate subcategories (such as "Science" or "Biography"), which are pulled
from the list structure of the document.

For some language pairs and user communities, it may be appropriate to import
an entire Wikipedia into Guampa \footnote{For example, as of this writing there are fewer than three thousand articles in the Guarani Wikipedia.} for
translation into another language. This approach is also supported.

We also include a script for easy export of bitext sentences; since the system
keeps an internal representation of sentence boundaries in the original
documents, it is easy to export one sentence per line into the output files. To
train an MT system, one would then run the appropriate preprocessing and
training pipeline on these exported files. In the near term, we also plan to
add export features in HTML and MediaWiki markup, for ease of publishing the
translated documents and adding them to the target-language Wikipedia.


\section{Implementation}
Guampa is a web application, with a user interface made of the AngularJS
\cite{angularjs} and Bootstrap \cite{bootstrap} toolkits. We internationalize
the user interface with the angular-translate package \cite{angular-translate}.
The server side of the application is implemented in Python 3, using the Flask
micro-framework \cite{flask}, SQLAlchemy
\cite{sqlalchemy} for object-relational mapping, and SQLite \cite{sqlite} as a
database. SQLite could easily be replaced with a more industrial-scale
database, should the need arise.

\section{Conclusions and Future Work}
In collaboration with language learners and activists in Paraguay, we will use
Guampa to build a bitext corpus for Spanish-Guarani for our own continuing MT
work with that language pair. This resource will be made publicly available on
our website, along with Guampa and our other software.

As we continue development of Guampa, with feedback and suggestions from our
users, we plan to add additional features, including recognition for prolific
translators, similar to Wikipedia's Barnstars. We would like to develop other
game-like features, including a feature to send periodic translation tasks by
email, so that users can be reminded to practice daily. Additionally, we will
implement exporting over the web and into formats other than plain text -- such
as TMX, Mediawiki markup, and HTML --  to facilitate reuse of the collected
corpora.

More technically ambitious and longer-range future features will include lookup
in a translation memory, with pluggable morphological analysis, and integrated
suggestions from machine translation. Our long-term goal is for Guampa to
become a collaborative computer-assisted translation tool.

Guampa is free software, released under the GPLv3 and available on GitHub at:
\\
\url{http://github.com/hltdi/guampa} , with a live demo server linked from that
site. It is under active development, but is already relatively easy to install
and adapt the particularities of different language pairs and the needs of
different translation communities. We welcome suggestions, bug reports,
questions, doubts, and development collaborators!

\bibliographystyle{lrec2006}
\bibliography{guampa}

\end{document}

